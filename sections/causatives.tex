
\chapter{Causative and Benefactive}
Ampari has three productive causative derivation suffixes \emph{-tV}/\emph{-dV}/\emph{-ntV}/\emph{-ndV}, \emph {-gV} and \emph {-mV} typical for most Dogon languages. Their distribution is lexically defined, though the input stems in each case do share common semantic features. Among the three, causative in \emph {-mV} is the most productive one. Morphotactically, \emph {-tV} and \emph {-gV} occupy the same linear position as the reversive (\emph{-rV}) and the inchoative (\emph {-yV}), while \emph {-mV} can be combined with those suffixes.

\section{Causatives in  \emph{-tV}/\emph{-dV}/\emph{-ntV}/\emph{-ndV}}\label{causT}
Sinchronically, the choice between 4 causative allomorphs (\emph{-tV}, \emph{-dV}, \emph{-ntV} and \emph{-ndV}) appears be unexplained by any synchronic phonotactic rule. These causatives are typical for inchoative-causative pairs discussed in Subsection \ref{subsection:inch_caus_sec}. Cf. examples (\getref{inchtr002} -- \getref{inchtr030}).  


\ex[aboveglftskip = 0 ex]<inchtr002>
\begingl
\gla m̀-bí-yé//
\glb 1SG-lie.down-INCH.PFV//
\glft \lq  I lied down\rq.//
\endgl
\xe 


\ex[aboveglftskip = 0 ex]<inchtr005>
\begingl
\gla m̀-bí-dé//
\glb 1SG-lie.down-CAUS.PFV//
\glft \lq  I have (sb) lie down\rq.//
\endgl
\xe 

\ex[aboveglftskip = 0 ex]<inchtr029>
\begingl
\gla Hárúnà dì-yɛ́//
\glb Harouna wash-INCH.PFV//
\glft \lq  Harouna washed himself\rq.//
\endgl
\xe 

\ex[aboveglftskip = 0 ex]<inchtr030>
\begingl
\gla Hárúnà dì-dɛ́//
\glb Harouna wash-CAUS.PFV//
\glft \lq  Harouna washed sb\rq.//
\endgl
\xe 


